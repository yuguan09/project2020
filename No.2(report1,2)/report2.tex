\documentclass[xelatex,ja=standard,jafont=noto]{bxjsarticle}
\usepackage[utf8]{inputenc}

\usepackage{amsmath}
\usepackage{amsthm}
\usepackage{amssymb}
\usepackage{mathrsfs}
\usepackage{graphicx} 

\usepackage{tikz}
\usetikzlibrary{shapes,arrows}
\usepackage{verbatim}
\tikzstyle{block} = [draw, fill=white, rectangle, 
    minimum height=3em, minimum width=6em]
\tikzstyle{sum} = [draw, fill=white, circle, node distance=1cm]
\tikzstyle{input} = [coordinate]
\tikzstyle{output} = [coordinate]
\tikzstyle{pinstyle} = [pin edge={to-,thin,black}]



\newtheorem{theorem}{Theorem}
\newtheorem{corollary}{Corollary}
\newtheorem{lemma}{Lemma}
\newtheorem{example}{Ex\documentclass{article}
	\usepackage{CJKutf8}
	\usepackage{amsmath}
	\usepackage{amsthm}
	\usepackage{amssymb}
	ample}
\newtheorem{definition}{Definition}

\def\ds{\displaystyle}
\def\ul{\underline}
	\title{制御工学3レポート2	}
	\author{BQ18026,関宇 }
	\date{Oct.20,2020}
	
	
\begin{document}
		\maketitle
		
		
		
	\section{次の行列に対する状態遷移行列を計算せよ.また,これらの t -> $\infty$ における
極限値と,その行列の固有値の関係について考察せよ.}

a)

\begin{equation}
    {
    \left[
    \begin{array}{cc}
        0 & 0 \\
        0 & 0
    \end{array}
    \right]
    }
\end{equation}
	
	b)式より、
	
	\begin{equation}
	    \frac{d}{dt}e^{At}=Ae^{At}=e^{At}A
	\end{equation}
	
	ここで、三次元の行列指数関数を考える.
	
	\begin{equation}
	    e^{At}={
\left[ \begin{array}{ccc}
e^{a_{1}t}&&\\
&e^{a_{2}t}&\\
&&e^{a_{3}t}
\end{array}
\right ]},A={
\left[ \begin{array}{ccc}
a_{1}t&&\\
&a_{2}t&\\
&&a_{3}t
\end{array}
\right ]}
	\end{equation}
	
	
	\begin{equation}
	    \frac{d}{df}e^{At}={
\left[ \begin{array}{ccc}
a_{1}e^{a_{1}t}&&\\
&a_{2}e^{a_{2}t}&\\
&&a_{3}e^{a_{3}t}
\end{array}
\right ]}
	\end{equation}
	
	\begin{equation}
	    ={
\left[ \begin{array}{ccc}
a_{1}t&&\\
&a_{2}t&\\
&&a_{3}t
\end{array}
\right ]}{
\left[ \begin{array}{ccc}
e^{a_{1}t}&&\\
&e^{a_{2}t}&\\
&&e^{a_{3}t}
\end{array}
\right ]}=Ae^{At}
	\end{equation}
	
	特に
	
	\begin{equation}
	    e^{At}A={
\left[ \begin{array}{ccc}
e^{a_{1}t}&&\\
&e^{a_{2}t}&\\
&&e^{a_{3}t}
\end{array}
\right ]}{
\left[ \begin{array}{ccc}
a_{1}t&&\\
&a_{2}t&\\
&&a_{3}t
\end{array}
\right ]}={
\left[ \begin{array}{ccc}
a_{1}e^{a_{1}t}&&\\
&a_{2}e^{a_{2}t}&\\
&&a_{3}e^{a_{3}t}
\end{array}
\right ]}
	\end{equation}
	
	よって、性質bが証明できる。
	
	
	




	
	
\end{document}