\documentclass[xelatex,ja=standard,jafont=noto]{bxjsarticle}
\usepackage[utf8]{inputenc}

\usepackage{amsmath}
\usepackage{amsthm}
\usepackage{amssymb}
\usepackage{mathrsfs}
\usepackage{graphicx} 

\usepackage{tikz}
\usetikzlibrary{shapes,arrows}
\usepackage{verbatim}
\tikzstyle{block} = [draw, fill=white, rectangle, 
    minimum height=3em, minimum width=6em]
\tikzstyle{sum} = [draw, fill=white, circle, node distance=1cm]
\tikzstyle{input} = [coordinate]
\tikzstyle{output} = [coordinate]
\tikzstyle{pinstyle} = [pin edge={to-,thin,black}]



\newtheorem{theorem}{Theorem}
\newtheorem{corollary}{Corollary}
\newtheorem{lemma}{Lemma}
\newtheorem{example}{Ex\documentclass{article}
	\usepackage{CJKutf8}
	\usepackage{amsmath}
	\usepackage{amsthm}
	\usepackage{amssymb}
	ample}
\newtheorem{definition}{Definition}

\def\ds{\displaystyle}
\def\ul{\underline}
	\title{計算レポート4	}
	\author{bq18026\\YBIOYO\footnote{ヴィジュネル暗号を使った} }
	\date{11.8,2020}
	
	
\begin{document}
\maketitle

\section{相対論的な運動エネルギーの導出}

相対論的運動方程式は
\begin{equation}
    \mathbb{F}=\frac{d}{dt}(m(v)\mathbb{V})
\end{equation}

ただし

\begin{equation}
    m(v)=\frac{m_{0}}{\sqrt{1-(\frac{v}{c})^{2}}}
\end{equation}

微小な仕事は力と変位ベクトルの内積で、結果的にKという量を表すことができる.

\begin{equation}
    dW=\mathbb{F}\cdot\mathbb{R}=dK
\end{equation}

両辺をdtを割ると

\begin{equation}
    \frac{dW}{dt}=\frac{\mathbb{F}\cdot\mathbb{R}}{dt}=\frac{dK}{dt}
\end{equation}

続いて式1を式4に代入して

\begin{equation}
    \frac{dW}{dt}=\frac{d}{dt}(m(v)\mathbb{V})\cdot\mathbb{V}=(m^{'}(v)\mathbb{V}+m(v)\mathbb{V}^{'})\cdot\mathbb{V}
\end{equation}

\begin{equation}
    =m^{'}(v)\mathbb{V}\cdot\mathbb{V}+m(v)\mathbb{V}^{'}\cdot\mathbb{V}=m^{'}(v)v^{2}+m(v)v^{'}v
\end{equation}

m(v)について、

\begin{equation}
    m^{'}(v)=m_{0}\frac{vv^{'}}{c^{2}}(1-(\frac{v}{c})^{2})^{-\frac{3}{2}}
\end{equation}

これを式6に代入すると

\begin{equation}
    m_{0}\frac{v^{3}v^{'}}{c^{2}}(1-(\frac{v}{c})^{2})^{-\frac{3}{2}}+m_{0}vv^{'}(1-(\frac{v}{c})^{2})^{-\frac{1}{2}}
\end{equation}

\begin{equation}
=m_{0}vv^{'}(1-(\frac{v}{c})^{2})^{-\frac{3}{2}}[\frac{v^{2}}{c^{2}}+(1-(\frac{v}{c})^{2})]    
\end{equation}

\begin{equation}
    =m_{0}vv^{'}(1-(\frac{v}{c})^{2})^{-\frac{3}{2}}
\end{equation}

式4と式7によって、

\begin{equation}
    \frac{dK}{dt}=\frac{d}{dt}m(v)c^{2}
\end{equation}

なので、

\begin{equation}
    K=m(v)c^{2}
\end{equation}
\end{document}