\documentclass[xelatex,ja=standard,jafont=noto]{bxjsarticle}
\usepackage[utf8]{inputenc}

\usepackage{amsmath}
\usepackage{amsthm}
\usepackage{amssymb}
\usepackage{mathrsfs}
\usepackage{graphicx} 

\usepackage{tikz}
\usetikzlibrary{shapes,arrows}
\usepackage{verbatim}
\tikzstyle{block} = [draw, fill=white, rectangle, 
    minimum height=3em, minimum width=6em]
\tikzstyle{sum} = [draw, fill=white, circle, node distance=1cm]
\tikzstyle{input} = [coordinate]
\tikzstyle{output} = [coordinate]
\tikzstyle{pinstyle} = [pin edge={to-,thin,black}]



\newtheorem{theorem}{Theorem}
\newtheorem{corollary}{Corollary}
\newtheorem{lemma}{Lemma}
\newtheorem{example}{Ex\documentclass{article}
	\usepackage{CJKutf8}
	\usepackage{amsmath}
	\usepackage{amsthm}
	\usepackage{amssymb}
	ample}
\newtheorem{definition}{Definition}

\def\ds{\displaystyle}
\def\ul{\underline}
	\title{追加課題1}
	\author{bq18026\\001010001011110000010100001101000001100000000100\footnote{今年から木村研に所属するので、たくさん暗号に関することが勉強しています.氏名も適当に暗号化しました.}}
	\date{07.11,2020}
	
	
\begin{document}
\maketitle

\section{PID制御系}

\begin{figure}[h!]
    \centering
    \includegraphics[scale=0.8]{a.jpg}
    \caption{PID制御系}
\end{figure}

簡単のため、モータ系の伝達関数をP(s),PIDコントローラーの伝達関数をK(s)とする.\\

\subsection{}
素直な考え方、出力yは外乱入力dと目標値入力rにおけるそれぞれのフィードバック系の組み合わせによって構成されている.なので、

\begin{equation}
    y=G_{r}(s)r+G_{d}(s)
\end{equation}

\begin{equation}
    y=r\frac{K(s)P(s)}{1+K(s)P(s)}+d\frac{P(s)}{1+K(s)P(s)}
\end{equation}

\begin{equation}
    y=r\cdot\frac{(K_{P}+\frac{K_{I}}{s}+K_{D}s)\frac{K}{(s^{2}T+1)}}{1+\frac{(K_{P}+\frac{K_{I}}{s}+K_{D}s)K}{(s^{2}T+1)}}+w\cdot\frac{\frac{K}{s(Ts+1)}}{1+\frac{K(K_{P}+\frac{K_{I}}{s}+K_{D}s)}{s(Ts+1)}}
\end{equation}
	
	\begin{equation}
	    =\frac{KK_{P}+\frac{KK_{I}}{s}+KK_{D}s}{s^{2}T+s+KK_{P}+\frac{KK_{I}}{s}+KK_{D}s}\cdot r+\frac{K}{s^{2}T+s+KK_{P}+\frac{KK_{I}}{s}+KK_{D}s}\cdot d
	\end{equation}
	
	\begin{equation}
	    =\frac{s^{2}KK_{D}+sKK_{p}+KK_{I}}{s^{3}T+s^{2}(1+KK_{D})+sKK_{p}+KK_{I}}\cdot r+\frac{Ks}{s^{3}T+s^{2}(1+KK_{D})+sKK_{p}+KK_{I}}\cdot d
	\end{equation}
	
	
	\section{PD-I制御系}
	
	\begin{figure}[h!]
    \centering
    \includegraphics[scale=0.8]{b.jpg}
    \caption{PD-I制御系}
\end{figure}

(1) と同じように考えると、まず外乱dに対するフィードバック系の伝達関数G1(s)を求める.

\begin{equation}
    G_{1}(s)=\frac{\frac{K}{s(Ts+1)}}{1+\frac{KK_{p}s}{s(Ts+1)}}=\frac{K}{s^{2}T+(1+KK_{P})s}
\end{equation}

式2と同じく

\begin{equation}
    y=\frac{K(s)G_{1}(s)}{1+K(s)G_{1}(s)}\cdot r+\frac{G_{1}(s)}{1+K(s)G_{1}(s)}\cdot d
\end{equation}

\begin{equation}
    =\frac{(K_{P}+\frac{K_{I}}{s})\cdot\frac{K}{s^{2}T+(1+KK_{P})s}}{1+(K_{P}+\frac{K_{I}}{s})\cdot\frac{K}{s^{2}T+(1+KK_{P})s}}\cdot r+\frac{K}{1+(K_{P}+\frac{K_{I}}{s})\cdot\frac{K}{s^{2}T+(1+KK_{P})s}}\cdot d
\end{equation}

\begin{equation}
    =\frac{KK_{P}+\frac{KK_{I}}{s}}{s^{2}T+(1+KK_{P})s+KK_{P}+\frac{KK_{I}}{s}}\cdot r+\frac{Ks}{s^{2}T+(1+KK_{P})s+KK_{P}+\frac{KK_{I}}{s}}\cdot d
\end{equation}\\

最後に、分子と分母を共にsをかける.

\begin{equation}
    y=\frac{sKK_{p}+KK_{I}}{s^{3}T+s^{2}(1+KK_{D})+sKK_{p}+KK_{I}}\cdot r+\frac{Ks}{s^{3}T+s^{2}(1+KK_{D})+sKK_{p}+KK_{I}}\cdot d
\end{equation}

\newpage

\section{I-PD制御}

\begin{figure}[h!]
    \centering
    \includegraphics[scale=0.8]{c.jpg}
    \caption{I-PD制御}
\end{figure}

2とまった同じく手順で考える.まず外乱dに対するフィードバック系の伝達関数G1(s)を求める.

\begin{equation}
    G_{1}(s)=\frac{K}{s^{2}T+(KK_{D}+1)s+KK_{P}}
\end{equation}

式2と同じように

\begin{equation}
    y=\frac{K(s)G_{1}(s)}{1+K(s)G_{1}(s)}\cdot r+\frac{G_{1}(s)}{1+K(s)G_{1}(s)}\cdot d
    \end{equation}
    
\begin{equation}
    =\frac{\frac{K_{I}}{s}(\frac{K}{s^{2}T+(KK_{D}+1)s+KK_{P}})}{1+\frac{KK_{I}}{s(s^{2}T+(KK_{D}+1)s+KK_{P})}}\cdot r+\frac{\frac{K}{s^{2}T+(KK_{D}+1)s+KK_{P}}}{1+\frac{KK_{I}}{s(s^{2}T+(KK_{D}+1)s+KK_{P})}}\cdot d
\end{equation}

    
\begin{equation}
    y=\frac{KK_{I}}{s^{3}T+(KK_{D}+1)s^{2}+KK_{P}s+KK_{I}}\cdot r+\frac{Ks}{s^{3}T+(KK_{D}+1)s^{2}+KK_{P}s+KK_{I}}\cdot d
\end{equation}
		
	

\end{document}